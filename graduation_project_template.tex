% Copyright (c) 2017 Ongun Kanat <ongun.kanat@gmail.com>
% Permission is hereby granted, free of charge, to any person obtaining a copy of 
% this software and associated documentation files (the "Software"), to deal in 
% the Software without restriction, including without limitation the rights to 
% use, copy, modify, merge, publish, distribute, sublicense, and/or sell copies of 
% the Software, and to permit persons to whom the Software is furnished to do so, 
% subject to the following conditions:
% 
% The above copyright notice and this permission notice shall be included in all 
% copies or substantial portions of the Software.
% 
% THE SOFTWARE IS PROVIDED "AS IS", WITHOUT WARRANTY OF ANY KIND, EXPRESS OR 
% IMPLIED, INCLUDING BUT NOT LIMITED TO THE WARRANTIES OF MERCHANTABILITY, FITNESS 
% FOR A PARTICULAR PURPOSE AND NONINFRINGEMENT. IN NO EVENT SHALL THE AUTHORS OR 
% COPYRIGHT HOLDERS BE LIABLE FOR ANY CLAIM, DAMAGES OR OTHER LIABILITY, WHETHER 
% IN AN ACTION OF CONTRACT, TORT OR OTHERWISE, ARISING FROM, OUT OF OR IN 
% CONNECTION WITH THE SOFTWARE OR THE USE OR OTHER DEALINGS IN THE SOFTWARE.

% 12pt and ISO A4 paper with title page add notitlepage for otherwise
\documentclass[a4paper, 12pt, titlepage]{article}

% Margins and page size
\usepackage[a4paper,top=2.5cm,bottom=2.5cm,left=3.3cm,right=2.2cm]{geometry}

% Page headers are set to top right
\usepackage{fancyhdr}
\pagestyle{fancy}
\renewcommand{\footrulewidth}{0pt} % clear rulers
\renewcommand{\headrulewidth}{0pt}
\lhead{} % Empty left header
\rhead{\thepage} % Page number at the right header
\cfoot{} % Clear center of the footer

% Use American English for dates etc.
%\usepackage[american]{babel}
% If document is in Turkish then use
% \usepackage[turkish]{babel}
% or for both
% \usepackage[turkish,american]{babel}

% Indent at section beginnings
%\usepackage{indentfirst}

% utf-8 support
\usepackage[utf8]{inputenc}

% Graphics for PDFTeX
\usepackage[pdftex]{graphicx}

% Figure placement
\usepackage{float}

% An enumeration package for flexible enumeration
\usepackage{enumitem}

% For fitting tables into the page width
\usepackage{makecell}
\renewcommand{\theadalign}{cc} % Centering and at the middle
\renewcommand{\theadfont}{\bfseries} % Bold table headers

% Helvetica Sans-serif fonts
\usepackage{helvet}
\usepackage{sectsty}
\allsectionsfont{\normalfont\sffamily\bfseries}
\sectionfont{\fontsize{18pt}{21.6pt}\sffamily\bfseries}
\subsectionfont{\fontsize{16pt}{19.2pt}\sffamily\bfseries}
\subsubsectionfont{\fontsize{14pt}{16.8pt}\sffamily\bfseries}

% Courier monospace font
\usepackage{courier}

% Table of contents dot fill, sans serif header and name change
\usepackage{tocloft}
\renewcommand{\cftsecleader}{\cftdotfill{\cftdotsep}}
\tocloftpagestyle{fancy}
\renewcommand{\cfttoctitlefont}{\sffamily\Large\bfseries}
\setlength{\cftbeforesecskip}{6pt}
\renewcommand{\contentsname}{Table of Contents}

% Links, both local and external
\usepackage{hyperref}
\hypersetup{
	unicode=true,
	colorlinks=true,
	urlcolor=blue,
	citecolor=black,
	menucolor=black,
	linkcolor=black
}

% Figure captions are bold
\usepackage[labelfont=bf,font=sf]{caption}

% Pseudocode from algorithmicx package
\usepackage{algorithmicx}
\usepackage{algpseudocode}
\usepackage[section,boxed]{algorithm}
\captionsetup[algorithm]{labelfont=bf,font=sf,justification=centering,position=top}

% Listings for implemented code
\usepackage{listings}
\lstset{basicstyle=\ttfamily,frame=lines,tabsize=4}
\renewcommand{\lstlistingname}{Listing}

% A powerful math notation package
\usepackage{amsmath}

% Title, author and date info
\title{Graduation Project}
\author{Besim Ongun Kanat}
\date{June 2017}

\begin{document}
\numberwithin{figure}{section}
\numberwithin{table}{section}
\numberwithin{lstlisting}{section}

\begin{titlepage}
    \bfseries % Make all text bold in this environment
    \sffamily % Similarly select sans-serif font
	\begin{center}
		\LARGE{\textbf{İSTANBUL TEKNİK ÜNİVERSİTESİ BİLGİSAYAR VE BİLİŞİM FAKÜLTESİ} } \\
		\vspace{5.5cm}
		\LARGE{Otomatik Transfer Fonksiyonu Oluşturucu Robotik Sistem}  \\
		\vspace{4.5cm}
		\Large{Bitirme Projesi} \\
        \vspace{0.5cm}
		\Large{Ali KADAM} \\
     	\Large{040100005} \\
        \vspace{4cm}
        \large{Bölüm: Bilgisayar Mühendisliği} \\
        \vspace{1.5cm}
        \large{Danışman: Asst. Prof. Dr. Gökhan İNCE} \\
		\vspace{\fill} % Fill out until the page end
		\large{\normalfont \sffamily June 2017}
	\end{center}
\end{titlepage}

\pagenumbering{Roman} % Capital roman numerals as page numbers
\newpage
\section*{Özgünlük Bildirisi}
\begin{enumerate}
    \item Bu çalışmada, başka kaynaklardan yapılan tüm alıntıların, ilgili kaynaklar \\ referans gösterilerek açıkça belirtildiğini,
    \item Alıntılar dışındaki bölümlerin, özellikle projenin ana konusunu oluşturan teorik çalışmaların ve yazılım/donanımın benim tarafımdan yapıldığını
    bildiririm.
\end{enumerate}
\vspace{1em}
İstanbul, 05.06.2017
\vspace{3em}\\Ali KADAM

\newpage
\section*{Önsöz}
Bu projede belirli bir alanı tarayarak belirlenen noktalardan ses sinyalleri gönderip bu sinyalleri işleyerek her noktaya ait transfer fonksiyonun hesaplayan bir robotik sistem yapılmıştır. Bu çalışmada hem robotik bir sistem geliştirilmiş, hemde ses sinyallerini işleyerek transfer fonksiyonları oluşturmuştur. Bu iki konu da iş hayatında çalışmak istediğim alanları kapsadığı için bu projenin gelecek hayatımda bana çok büyük faydaları olacağını düşünüyorum.


\newpage
\section*{Otomatik Transfer Fonkisyonu Oluşturucu Robotik Sistem}
\centerline{\large\bfseries (Özet)}
Robotik sistemler insan hayatının her aşamasında karşımıza çıkmaktadır. Günümüzde daha önce insanlar tarafından yapılan birçok işlem artık robotlar tarafından yapılmaktadır. Aynı zamanda insanlar tarafından yapılması mümkün olmayan işlemlerde robotlar tarafından yapılabilmektedir. Bu sebeple robotik teknolojilerin gelişmesi, teknolojinin diğer dallarının gelişmesi açısından da oldukça önemlidir.
Robot teknolojilerindeki en önemli gelişmelerden biride robotların dışarıdan gelen sinyalleri alıp işleyebilmesi olmuştur. Günümüzde robotlar gelen sesi yada görüntüyü işleyerek temel düzeyde görme, duyma gibi işlemleri gerçekleştirebilmektedir. Ama bu sinyaller her zaman mükemmel bir şeklide robota ulaşmayabilir. Örneğin aynı ses sinyali farklı noktalardan gönderildiğinde robotun algıladığı sinyal orijinal sinyalden farklı olacaktır. Aynı zamanda bütün noktalar içinde farklı olacaktır. Robotun düzgün bir şekilde sesi algılayabilmesi için bu farklılıkları anlayıp ona göre işlem yapması gerekmektedir.
Gönderilen ses sinyali ile algılanan sinyal arasındaki bağıntı transfer fonksiyonu ile ifade edilebilir. Eğer robot bir ortamdaki bütün noktalara ait transfer fonksiyonunu bilirse, daha sonra gelen ses sinyallerini işlerken bu fonksiyonları kullanabilir. Bu projede bir ortama ait farklı noktaların transfer fonksiyonunun hesaplanması amaçlanmıştır.
Bunun için iki düzlemde hareket imkanı sağlayan bir X-Y table tasarlanmıştır. Bu düzeneğe yerleştirilen bir speaker sayesinde sistemin daha önceden belirlenen noktalara giderek ses sinyali göndermesi sağlanmış ve gelen bu ses sinyalleri işlenerek her nokta için bir transfer fonksiyonu hesaplanmıştır. 

\newpage
\section*{Automatic Transfer Function Generator Robotic System}
\centerline{\large\bfseries (Summary)}
In this project, a robotic system has been developed to calculate transfer functions for points in a spesific area. This system has a XY table. This table carries a speaker to points we have decided before. Speaker plays a sound at each point. The system records this sounds and creates the transfer functions for all points. 


\newpage
\tableofcontents
\newpage

% For the ones who doesn't know: 1,2,..9 called West Arabic numbers
\pagenumbering{arabic}
\section{Giriş}
Robotların gelen sesleri algılayıp işleyebilmesi oldukça önemli bir konudur. Ama aynı ses sinyali her zaman aynı şeklide gelmez. Bu sinyal sesin geldiği nokta, çevre etkenleri gibi sebeplerle farklılıklar gösterebilir. Robot her defasında bu sinyali düzgün bir şekilde anlayabilmeli ve bazı durumlarda sinyalin geldiği noktayı tespit edebilmelidir. Bu işlemler transfer fonksiyonları kullanılarak gerçekleştirilebilir.
Kısaca tanımlamak gerekirse transfer fonksiyonu bir sistemin giriş ve çıkış bağıntıları arasındaki ilişkiyi tanımlayan fonksiyondur \cite {transfer function}. Bu fonksiyon aynı çevresel etkenler altında aynı değere sahip olacaktır. Bu şekilde bizim sistemimizdeki bir nokta için transfer fonksiyonunu belirlediğimiz zaman daha sonra bu noktadan gelen sinyali hem düzgün bir şekilde algılayabileceğiz, hem de sinyalin hangi noktadan geldiğini anlayabileceğiz.
Bu konuda daha önce Honda ve Kyoto Üniversitesinin birlikte gerçekleştirdiği HARK projesi[2] önemli bir referans noktası oluşturmaktadır. Bu projede kurulmuş mikrofon sisteminin etrafında belirlenen noktalardan sırasıyla ses sinyalleri gönderilmiştir. Daha sonra bu sinyaller kullanılarak farklı yönler için transfer fonksiyonu bulunması sağlanmıştır [3].
Bizim projemizdeki en büyük farklılık ise sistemin bilgisayar ortamında belirlenen noktalara kendisi giderek gerekli işlemeleri gerçeklemesidir. Sadece alanın ölçüleri ve ölçüm yapılmak istenilen noktalar arasındaki mesafe girilerek sistem çalıştırılabilir. Daha sonra sistem otomatik olarak hesapladığı noktalara giderek ses çalma ve kaydetme işlemlerini gerçekleştirecektir.
\newpage
\section{Projenin Tanımı ve Planı}
\subsection{Projenin Tanımı}
Bu projemiz 3 ana kısımdan oluşmaktadır. Birinci kısım XY table dizaynı ve programlanmasıdır. Bu kısımda iki boyutta hareket imkanı sağlayan bir XY table tasarlanmış ve gerçeklenmiştir. XY table iki motor tarafından kontrol edilmektedir. 
\begin{figure}[H]
    \centering
    \includegraphics[width=\linewidth]{gantt_diagram}
    \caption{The Gantt diagram of the project}
    \label{fig:ganttdiagram}
\end{figure}

\newpage
\section{Background}
A long long section with many papers.

\newpage
\section{Analysis and Modeling}
After long hours of studying we decided. A* is the best algorithm for searching on the graphs.

\newpage
\section{Design and Implementation}
We implemented the depth first search algorithm as shown in Algorithm \ref{algo:dfs}.
\begin{algorithm}[H]
    \caption{The depth first search algorithm}
    \label{algo:dfs}
    \begin{algorithmic}[1]
        \State \textbf{Graph} $G$
        \State \textbf{Node} $start$
        \Function{Depth-First-Search}{$G$, $start$}
        \State \textbf{Tree} $ T $ \Comment The resulting search tree
        \State \textbf{Stack} $ S $ \Comment An empty stack
        \State \textbf{Set} $ V $ \Comment An empty set of visited nodes
        \State \Call{set-root}{$ T $,$ current $}
        \State \Call{push}{$ S $,$ start $}
        \While{\Call{not-empty}{S}}
        \State $current \gets$ \Call{pop}{$ S $}
        \If{\textbf{not} \Call{Contains}{$ V $, $ current $} }
        \State \Call{insert}{$ V $, $ current $}
        \ForAll{$ n $ : \Call{neighbors}{$ current $} }
        \State \Call{push}{$ S $, $ n $}
        \State \Call{insert-sub-node}{$ T $, $ current $, $ n $}
        \Comment Insert node to subtree of $ current $
        \EndFor
        \EndIf
        \EndWhile
        \State \Return $ T $
        \EndFunction
    \end{algorithmic}
\end{algorithm}

\newpage
\section{Testing and Evaluation}
We compared the performance of the graph search algorithms. The results can be seen in the Table \ref{tbl:results}.

\begin{table}[H]
    \caption{Performance test results of the graph search algorithms}
    \label{tbl:results}
    \centering
    \begin{tabular}{|c|r|r|r|} 
        \hline 
        \thead{Algorithm} & \thead{Number of \\ Generated Nodes} & \thead{Number of \\ Nodes Expanded} & \thead{Max Number of \\ Nodes in The Frontier} \\ 
        \hline 
        BFS &  77480 & 6340 & 71142 \\ 
        \hline 
        DFS & 820 & 88 & 760 \\ 
        \hline 
        A* & 376 & 33 & 338 \\ 
        \hline 
    \end{tabular}
\end{table}

\newpage
\section{Conclusion and Future Work}
We tested algorithms and decided that future is bright.

\newpage
\bibliographystyle{IEEEtran}
\bibliography{references.bib} 

\end{document}
